\documentclass[11pt,a4paper]{moderncv}
\moderncvtheme[blue]{classic}
\usepackage[utf8]{inputenc}
\usepackage[scale=.6,margin=.5in,top=.3in]{geometry}
\usepackage{multicol}
\AtBeginDocument{\setlength{\maketitlenamewidth}{6cm}}
\AtBeginDocument{\recomputelengths}

\AfterPreamble{
    \hypersetup{
      pdfauthor={Stefan Eng},
      pdftitle={Stefan Eng Resume},
      pdfsubject={Stefan Eng Resume},
      colorlinks,
      breaklinks,
      urlcolor=blue
    }
}

% personal data
\firstname{Stefan}
\familyname{Eng}
\phone{(818) 331-5307}
\email{stefaneng13@gmail.com}
\address{3517 Brookhill Street}{La Crescenta, CA 91214}
\extrainfo{\href{https://github.com/stefaneng}{github.com/stefaneng}}


\nopagenumbers{}

\begin{document}
\maketitle

\section{Education}
\cventry{}{Mathematics B.S. \& Computer Science B.S., Expected Spring 2016}{\newline California State University}{Northridge}{GPA: \textbf{3.51}}{}
%\cventry{2007--2011}{High School Diploma}{\newline Crescenta Valley High School}{La Crescenta}{}{}

\section{Experience}

\cventry{April 2014 - Current}{Data Scientist}{Caltech/JPL}{Pasadena}{}{
  \begin{itemize}
  \item Manage data processing pipeline for Cassini and Dawn missions which includes Scala/Apache Spark ingest code and managing Elasticsearch cluster.
  \item Worked on an API that provides a common interface to mission's telemetry data which supports data visualization dashboards.
  \item Wrote many parsers to bring old proprietary data formats into common and modern formats. Included XML parsing, web scraping, Scala's parser combinators for tricky data formats.
  \item Manage docker deployments to Apache Mesos, using Marathon, for visualization dashboard project.
  \item Lead a project to extract data, using Apache Tika, from word documents into MongoDB. Data was then displayed through a web application, written in AngularJS, to allow easy searching and downloading.

%   \item Developed a scraping program to archive a project's data. Addition scripts were developed to allow easy access to data.
    % \item Scraped patent data and images for a visualization team to showcase JPL's patents.

  \end{itemize}
}

\cventry{Sep 2013 - April 2014}{Software Developer}{Caltech/JPL}{Pasadena}{}{
%\textit{Academic Part Time} - Software developer for the Microwave Limb Sounder(MLS) mission.
\begin{itemize}
\item Processed atmospheric data with Python and IDL.
\item Maintained and improved the MLS website with PHP and Javascript.
\item Automated repetitive tasks in data processing pipeline with bash scripts.
\item Set up and administered an SVN server, served through Apache.
\end{itemize}
}

\cventry{June 2013 - Aug 2013}{Data Scientist, Intern}{Caltech/JPL}{Pasadena}{}{
%\textit{Intern} -
\begin{itemize}
\item Assisted different projects and created many high visibility prototypes to help describe and visualize data to supplement research.
\item Used cloud technologies such as Amazon AWS to store and process data.
\item Worked with R, Python, Perl, for data analysis and visualizations.
\item Wrote a Node.js web application using Express.js to help Network Engineers better view and manage their data.
\end{itemize}
}

%\cventry{Mar 2012 - Nov 2013}{Web Administrator}{Crescenta Valley High School Football Team}{La Crescenta}{}{
%\textit{Part Time} -
%\begin{itemize}
%\item Managed updates for the Crescenta Valley High School football team.
%\item Typical in-season work included adding scores to the website and updat%ing rosters.
%\end{itemize}
%}

\section{Programming Languages}
\cvline{}{\small Python, Javascript, R, Java, Scala}
\section{Tools}
\cvline{}{\small Linux, \LaTeX, Git, SVN, Emacs}

\section{Projects}
\cvline{}{\textbf{CamAcc (Camera for the visually impaired)} \vfill
\begin{itemize}
\item Android camera app that is controlled through voice commands to help the visually impaired take photos
\item 1st Place in 2014 SS12 Code for a Cause Competition
\item Features included face detection for one, two or multiple people, detecting off-centered faces, level detection, sharing photos through social media, and applying filters
\item Play store link: \href{https://play.google.com/store/apps/details?id=com.camacc}{https://play.google.com/store/apps/details?id=com.camacc}
\end{itemize}
\vspace{-5mm}
}
\cvline{}{\textbf{Lecture Sync}
  \begin{itemize}
  \item Web application and Android app to help students recreate a lecture by pairing audio, notes, and slides in sync
  \item Web application was built with Node.js, MongoDB with Mongoose, hosted on an Amazon AWS EC2.
  \end{itemize}
}
%\cvline{}{\textbf{Collaborative Gaming - HackMIT.}
%\vfill
%Prototyped the idea of a collaborative HTML5 game where multiple users can attempt to control a player. Implemented using Firebase and Javascript with HTML5.}
\cvline{}{\textbf{More projects at} \href{https://github.com/stefaneng}{github.com/stefaneng}}

%\section{Professional Societies}
%\cvline{}{IEEE and IEEE Computer Society}
%\cvline{}{Association for Computing Machinery(ACM)}%

%\section{Courses}
%\cvline{Computer Science}{}
%\cvline{Mathematics}{}

\end{document}